%----------------------------------------------------------------------------------------
%	PACKAGES AND OTHER DOCUMENT CONFIGURATIONS
%----------------------------------------------------------------------------------------

\documentclass[12pt]{article}

\begin{document}

\begin{titlepage}

\newcommand{\HRule}{\rule{\linewidth}{0.5mm}} % Defines a new command for the horizontal lines, change thickness here

\center % Center everything on the page
 
%----------------------------------------------------------------------------------------
%	HEADING SECTIONS
%----------------------------------------------------------------------------------------

\textsc{\LARGE Royal Institute of Technology}\\[1.5cm] % Name of your university/college
\textsc{\Large Oppostion Report}\\[0.5cm] % Major heading such as course name

%----------------------------------------------------------------------------------------
%	TITLE SECTION
%----------------------------------------------------------------------------------------

\HRule \\[0.8cm]
{\Large \bfseries Efficient and Reliable Filesystem Snapshot Distribution }\\[0.4cm] % Title of your document
\HRule \\[1.5cm]
 
%----------------------------------------------------------------------------------------
%	AUTHOR SECTION
%----------------------------------------------------------------------------------------

\begin{minipage}{0.4\textwidth}
\begin{flushleft} \large
\emph{Author:}\\
Lauri \textsc{VOSANDI} % Your name
\end{flushleft}
\end{minipage}
~
\begin{minipage}{0.4\textwidth}
\begin{flushright} \large
\emph{Opponent:} \\
Abhimanyu \textsc{Babbar}\\ % Supervisor's Name
babbar@kth.se

\end{flushright}
\end{minipage}\\[4cm]

% If you don't want a supervisor, uncomment the two lines below and remove the section above
%\Large \emph{Author:}\\
%John \textsc{Smith}\\[3cm] % Your name

%----------------------------------------------------------------------------------------
%	DATE SECTION
%----------------------------------------------------------------------------------------

{\large \today}\\[3cm] % Date, change the \today to a set date if you want to be precise

%----------------------------------------------------------------------------------------
%	LOGO SECTION
%----------------------------------------------------------------------------------------

%\includegraphics{Logo}\\[1cm] % Include a department/university logo - this will require the graphicx package
 
%----------------------------------------------------------------------------------------

\vfill % Fill the rest of the page with whitespace

\end{titlepage}


\section{Report Title Justification}

The report title \textit{"Efficient and Reliable Filesystem Snapshot Distribution"} is completely justified as the report mainly talks about Provisioning System which works by snapshotting the filesystem and then distributing it to the nodes over the network. 

\section{Project Contribution}

The contribution of the project was easy to understand as the author has clearly stated it as 3 points in the \textit{Contribution} section in the report.

\section{Strong Points}

\begin{enumerate}

\item The author in this report has clearly stated the Problem definition which is the error prone and time consuming nature of the deployment of Linux environment. In addition to this, the motivation for the thesis and aim are clearly stated. 

\item The solution provided by the author works in conjunction to the already existing Configuration management techniques like Puppet and Salt. In addition to this, the solution has already been used and appreciated by IT support and other institutions which further bolster the motivation behind the thesis and the applicability of solution. 

\item In order to help with the comprehension regarding content in the Design and Architecture, the author has successfully tried to explain it using figures and self defined examples. The figures used are simple and easy to understand. 

\item The author at the end of the report along with clearly defining the future work associated with the project, reaffirmed the scope of the project and the tasks mainly completed as part of thesis. It helps to clearly define the boundaries of the project.

\end{enumerate}


\section{Weak Points}

\begin{enumerate}

\item At some places in the report it is difficult to distinguish between the user's own opinion and the information used from the literature during background study.

\item The diagrams constructed as part of evaluation has missing information about the data represented on the axis, which makes it difficult to understand the diagram in isolation.

\item At certain places, the author has failed to provide references for the information presented which makes it hard to comprehend it through further lookup.

\end{enumerate}


\section{Language and Technical Performance}

Apart from few grammatical mistakes and typos, the report is written in a clear language with logical relation between the various sections in the report.


\section{References}
The references provided by the author in the report have a consistent format and useful for lookup to further information for better comprehension of the report.

\end{document}